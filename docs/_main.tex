\documentclass[justified]{tufte-book}

% ams
\usepackage{amssymb,amsmath}

\usepackage{ifxetex,ifluatex}
\usepackage{fixltx2e} % provides \textsubscript
\ifnum 0\ifxetex 1\fi\ifluatex 1\fi=0 % if pdftex
  \usepackage[T1]{fontenc}
  \usepackage[utf8]{inputenc}
\else % if luatex or xelatex
  \makeatletter
  \@ifpackageloaded{fontspec}{}{\usepackage{fontspec}}
  \makeatother
  \defaultfontfeatures{Ligatures=TeX,Scale=MatchLowercase}
  \makeatletter
  \@ifpackageloaded{soul}{
     \renewcommand\allcapsspacing[1]{{\addfontfeature{LetterSpace=15}#1}}
     \renewcommand\smallcapsspacing[1]{{\addfontfeature{LetterSpace=10}#1}}
   }{}
  \makeatother

\fi

% graphix
\usepackage{graphicx}
\setkeys{Gin}{width=\linewidth,totalheight=\textheight,keepaspectratio}

% booktabs
\usepackage{booktabs}

% url
\usepackage{url}

% hyperref
\usepackage{hyperref}

% units.
\usepackage{units}


\setcounter{secnumdepth}{2}

% citations

% pandoc syntax highlighting

% longtable
\usepackage{longtable,booktabs}

% multiplecol
\usepackage{multicol}

% strikeout
\usepackage[normalem]{ulem}

% morefloats
\usepackage{morefloats}


% tightlist macro required by pandoc >= 1.14
\providecommand{\tightlist}{%
  \setlength{\itemsep}{0pt}\setlength{\parskip}{0pt}}

% title / author / date
\title{Philosophical Ethics}
\author{George Matthews}
\date{}

%\usepackage{MinionPro}
%\usepackage{fontspec}
%\newfontfamily\DejaSans{DejaVu Sans}

\newcommand{\given}{\mid}
\renewcommand{\neg}{\mathbin{\sim}}
\renewcommand{\wedge}{\mathbin{\&}}
\renewcommand{\u}{U}
\newcommand{\gt}{>}
\newcommand{\p}{Pr}
\newcommand{\E}{E}
\newcommand{\EU}{EU}
\newcommand{\pr}{Pr}
\newcommand{\po}{Pr^*}
\definecolor{bookred}{RGB}{228,6,19}
\definecolor{bookblue}{RGB}{0,92,169}
\definecolor{bookpurple}{RGB}{114,49,94}

\newenvironment{epigraph}%
{
\begin{flushright}
\begin{minipage}{20em}
\begin{flushright}
\itshape
}%
{
\end{flushright}
\end{minipage}
\end{flushright}
}
\newenvironment{problem}{\begin{quote}\normalsize}{\end{quote}}
\newenvironment{puzzle}{\begin{quote}\normalsize}{\end{quote}}
\newenvironment{argument}{\begin{quote}\normalsize}{\end{quote}}
\usepackage{fontawesome}
\newenvironment{warning}{\begin{itemize}\item[\faBan]}{\end{itemize}}
\usepackage{marvosym}
\newenvironment{info}{\begin{itemize}\item[\Info]}{\end{itemize}}

%%%% Kevin Godny's code for title page and contents from https://groups.google.com/forum/#!topic/tufte-latex/ujdzrktC1BQ
\makeatletter
\renewcommand{\maketitlepage}{%
\begingroup%
\setlength{\parindent}{0pt}
{\fontsize{18}{18}\selectfont\textit{\@author}\par}
\vspace{1.75in}{\fontsize{36}{14}\selectfont\@title\par}
\vspace{0.5in}{\fontsize{20}{14}\selectfont A Guidebook for Beginners\par}
\vspace{0.5in}{\fontsize{14}{14}\selectfont\textsf{\smallcaps{v0.1 beta}}\par}
\vfill{\fontsize{12}{12}\selectfont\textit{A Creative Commons publication: License CC BY-SA 4.0}\par}
\thispagestyle{empty}
\endgroup
}
\makeatother

% Change shape from [display] to [block] to keep chapter numbers and titles on the same line
\titleformat{\chapter}%
  [block]% shape
  {\relax\ifthenelse{\NOT\boolean{@tufte@symmetric}}{\begin{fullwidth}}{}}% format applied to label+text
  {\itshape\huge\thechapter}% label
  {3em}% horizontal separation between label and title body
  {\huge\rmfamily\itshape}% before the title body
  [\ifthenelse{\NOT\boolean{@tufte@symmetric}}{\end{fullwidth}}{}]% after the title body


\usepackage{etoolbox}
% Jesse Rosenthal's code from https://groups.google.com/forum/#!topic/pandoc-discuss/wCF78X6SvwY
% Avoid new pagraph/indent after lists, quotes, etc.
\makeatletter
\newcommand{\gobblepars}{%
    \@ifnextchar\par%
        {\expandafter\gobblepars\@gobble}%
        {}}
\newcommand{\eatpar}{\@ifnextchar\par{\@gobble}{}}
\newcommand{\forcepar}{\par}
\makeatother
\AfterEndEnvironment{quote}{\expandafter\gobblepars}
\AfterEndEnvironment{enumerate}{\expandafter\gobblepars}
\AfterEndEnvironment{itemize}{\expandafter\gobblepars}
\AfterEndEnvironment{description}{\expandafter\gobblepars}
\AfterEndEnvironment{example}{\expandafter\gobblepars}
\AfterEndEnvironment{argument}{\expandafter\gobblepars}
\AfterEndEnvironment{problem}{\expandafter\gobblepars}
\AfterEndEnvironment{info}{\expandafter\gobblepars}
\AfterEndEnvironment{warning}{\expandafter\gobblepars}
\AfterEndEnvironment{longtable}{\expandafter\gobblepars} % not working, why?


% prevent extra space when \newthought follows \section
% see: https://tex.stackexchange.com/questions/291746/tufte-latex-newthought-after-section
\makeatletter
\def\tuftebreak{%
  \if@nobreak\else
    \par
    \ifdim\lastskip<\tufteskipamount
      \removelastskip \penalty -100
      \tufteskip
    \fi
  \fi
}
\makeatother

% indent lists a bit
\usepackage{enumitem}
\setlist[1]{leftmargin=24pt}

\begin{document}

\maketitle



{
\setcounter{tocdepth}{1}
\tableofcontents
}

\hypertarget{preface}{%
\chapter*{Preface}\label{preface}}
\addcontentsline{toc}{chapter}{Preface}

\newthought{This} text is my attempt to scratch two itches. The first is wanting a presentation of the major philosophical approaches to ethics that I can actually agree with. Teaching ethics over the years has been a process of active exploration for me as I have come to a more nuanced and richer appreciation of what ethical thinking and theorizing is all about, as well as some ideas about how I think the main strands of argument should play out. Yes this is a partisan effort, but it's all subject to revision and refinement based on, I hope at least, the better argument. That's what I am trying to get across here. The second itch I am trying to scratch here has to do with initiatives in open education, and I'd like this text to contribute in its own small way to the much larger and more influential open source movement and philosophy of which I consider it a part. Knowledge is only ours to share. Yes of course writers, developers and publishers do hard work that deserves compensation. But intellectual property is a false idol that deserves to be smashed. So here is my effort to chip away at it --- knowledge should free us and and not sink us into both literal and figurative debt.

\href{https://github.com/gwmatthews/ethics/raw/master/docs/_main.pdf}{Printable pdf edition available here}.

\hypertarget{acknowledgements}{%
\section*{Acknowledgements}\label{acknowledgements}}
\addcontentsline{toc}{section}{Acknowledgements}

\newthought{This} book would not be possible without the work of numerous people who make and share amazing work in the open source software community. It is based in particular on the work of \href{https://github.com/yihui}{Yi Hui} and the other developers of \href{https://github.com/rstudio/bookdown}{bookdown} and \href{https://rstudio.com/products/rstudio/}{Rstudio}, as well as \href{https://github.com/jweisber}{Jonathan Weisberg} whose restyling of the Tufte theme for bookdown books is both stylish and functional.

\textbf{License CC BY-SA 4.0}

The book is released under a creative commons \href{https://creativecommons.org/licenses/by-sa/4.0/}{CC BY-SA 4.0} license. This means that this book can be reused, remixed, retained, revised and redistributed (including commercially) as long as appropriate credit is given to the authors. If you remix, or modify the original version of this open textbook, you must redistribute all versions of this open textbook under the same license - CC BY-SA 4.0.

\hypertarget{part-some-preliminaries}{%
\part*{Some Preliminaries}\label{part-some-preliminaries}}
\addcontentsline{toc}{part}{Some Preliminaries}

\hypertarget{the-examined-life}{%
\chapter{The Examined Life}\label{the-examined-life}}

\begin{epigraph}
\ldots{}the unexamined life is not worth living.\\
---Socrates
\end{epigraph}

\newthought{We humans} are endowed with a unique capacity, the ability to reflect on what we believe and do. Unlike other animals, we are capable of taking a distance from the evidence of our senses and asking ourselves, ``Should I trust what I see or not?'' Likewise in the realm of desire and action: we can examine our own desires, intentions and plans and ask ourselves, ``Should I act on these or not?'' In both cases we are capable of stepping back from the immediate demands of our situation and seeking orientation from another source - we seek \emph{reasons} to believe or doubt what we see and reasons to follow or resist our urges. This is worth dwelling on for a little while since it doesn't seem to me to be too much of an exaggeration to say that much of our knowledge of and power over the natural world as well as many of the dilemmas we face in acting in the world are made possible by the fact of our reflective capacity to respond to reasons.

\hypertarget{what-do-i-know}{%
\section{What do I know?}\label{what-do-i-know}}

\newthought{As mammals} outfitted with complex nervous systems we are constantly receiving input from our senses. We perceive colors, sounds, smells, tastes and bodily sensations from the moment we wake up until the moment we fall completely asleep each day and even in sleep we are not completely shut off from sensory input. All of this raw sensation is processed behind the scenes of our conscious minds in myriad ways we are just beginning to understand, and as a result we are presented with a picture of a world of objects in space and time interacting with each other and us in many different ways. This much we share with other animals, or at least with those with whom we have a cerebral cortex in common. But unlike those other animals, our unique linguistic capacity enables us to engage with the world of our experience in a way that they cannot.

First of all, our ability to use and understand language enables us to explicitly categorize and classify what we experience - things are not just there in our surroundings as objects we happen to come across. Instead we both consciously and unconsciously organize the things we encounter into groups based on concepts such as: edible or inedible, animate or inanimate, threatening or safe, members of our social group or not members of our group, male or female, cause or effect, and so on. Now although there is clear evidence that certain animals do this to a limited extent as well - dogs, for example distinguish between their owners and strangers quite readily and reliably - for us, these acts of categorizing and classifying can be endlessly expanded and modified and also made fully explicit to our own awareness. We can endlessly expand our categories to include anything and everything conceivable - ``80's hair bands,'' ``things not good for eating in bed,'' ``mops,'' and ``former presidents,'' just to give a quick random sample. Not only do we thus have an infinitely variable way of looking at the world and organizing our experience according to concepts and categories but we can see and understand how we are doing so since we can articulate this organization verbally. Hence we can recognize that we are dividing up the world in a particular way and do it in a different way. We can add new categories or modify how we use them as we notice new similarities or subtle distinctions among things. We may also revise and refine our categories as accumulated personal and shared experience reveals to us their strengths and weaknesses - whether they ``carve nature at its joints'' or not. This ability to look at things in new ways as a result of collectively accumulated experience is rooted in the fact that we use language to do so and language is both infinitely extensible and essentially shared with other humans. Most importantly for the story I am telling here, we can ask ourselves about the implications of the way we look at the world, and we can wonder about whether we have good reasons for looking at things as we do.

This is a point that it is hard to overemphasize but also easy to miss since we take it so much for granted. By asking ourselves about the reasons we have for believing that some aspect or other of our experience is true we are asking ourselves not only about the way things are, but about the way things \emph{should} be; not just what we happen to believe about things based on their appearance to us, but about what we \emph{should} believe about them because it reflects their true reality. And by asking ourselves such questions we are asking what philosophers call normative questions, questions that have to do with values, with concepts like right, wrong, good, bad, true, false, beautiful and ugly.

\hypertarget{what-should-i-do}{%
\section{What should I do?}\label{what-should-i-do}}

\newthought{Thus} far I have been emphasizing the role of reflection and the seeking of reasons in our attempts to understand the world in which we live. But this of course is an incomplete picture, since we are not just disembodied minds looking at and trying to figure out the world. We are embodied, social beings who feel and act on needs and impulses, experience emotions, form and try to realize intentions, coordinate or compete with others, and seek or shun each others' company. This practical side of human life is, just as we have seen for our experiential selves, equally capable of being made explicit and becoming an object of our reflective capacities. We don't have to simply act on whatever urges happen to come to our attention, we can stop and think about what to do instead. Hence we find ourselves presented with choices - should I follow my immediate urges, or should I refrain from doing so in order to realize other goals? Once again, just as in the case of belief, our reflective capacity introduces a normative dimension to human life as we come to ask ourselves questions about our own needs, desires and decisions. We wonder what we should do in some particular situation, perhaps when our feelings are telling us one thing and our experience is reminding us of the bad results the last time we acted on similar impulses. And this generalizes as well, as we come to reflect on our motivations as such, on which of our goals are more worth pursuing in the long run, on the nature of human motivation and goals in general, on what might truly be the best way to live our lives. And thus philosophical ethics is born as a product of reflection on our own decision making as potentially thoughtful social animals.

\hypertarget{philosophical-ethics}{%
\section{Philosophical Ethics}\label{philosophical-ethics}}

\newthought{Philosophical} ethics is nothing but the deliberate pursuit and clarification of this kind of reflection on our own values, actions and decisions. Even though, as I have been emphasizing, we all have the capacity to reflect on our lives and choices, we do not always spend the time or make the effort to do this carefully and deeply. This is because we are mostly preoccupied with the practical details of our own lives. We are too busy living to take the time to stop and think about the significance of what we are doing. However, at times in the lives of both individuals and societies the need to reflect more clearly on what we are doing becomes more of an imperative. For individuals the need to stop and think and to reconsider the basic assumptions on which we act often arises in relation to important life events or radical changes - the sudden loss of a loved one; the birth of a child; living through a natural disaster or a war; or even the transition to adulthood in which one assumes full moral and legal responsibility while also gaining the full rights and privileges accorded to adults. These are topics and situations, as we will see later, that are often the focus of discussions in the branch of philosophical ethics called applied ethics. In the case of societies, philosophical thinking in general and philosophical ethics in particular likewise flourish in times of great stress or change - for example when radically different societies suddenly make contact with each other; when new groups and ways of living displace old groups and ways; when new discoveries challenge peoples' basic views of the nature of things; when societies find their very existence threatened by seemingly insurmountable obstacles. In cases like these it becomes imperative to reflect more carefully on what we assume is of value to us individually and as a society, on what counts as a good life.

A philosophical approach to ethics, or moral philosophy, is concerned with a number of different sorts of questions. Thus the broader field of ethics can be divided up into a number of different regions or areas of concern. Some of the main questions and their corresponding sub-fields are:

\begin{marginfigure}
\textbf{Descriptive ethics}: what do people really think about right and
wrong?
\end{marginfigure}

\begin{itemize}
\tightlist
\item
  What ethical views do real people have? This is the concern of \textbf{descriptive ethics}, which tries to figure out what beliefs people happen to actually have concerning ethical questions. As such, descriptive ethics is not exclusively a philosophical approach to ethics in that sociologists, psychologists, anthropologists and other social scientists are also concerned with people's ethical beliefs in this sense.
\end{itemize}

\begin{marginfigure}
\textbf{Meta-ethics}: how does thinking about ethics work?
\end{marginfigure}

\begin{itemize}
\tightlist
\item
  What is the nature of ethical thinking and ethical concepts? This is usually referred to as \textbf{meta-ethics}, which refers to a higher-order or ``meta-level'' discussion about ethical modes of thinking. Here again, philosophers as well as social scientists often ask meta-ethical questions in their attempts to understand what is distinctive about ethical thinking as opposed to other modes of cognition.
\end{itemize}

\begin{marginfigure}
\textbf{Prescriptive ethics}: what is really the right thing to do?
\end{marginfigure}

\begin{itemize}
\tightlist
\item
  What ethical principles or decision-making procedures are really justified? This is the basic question addressed by normative or \textbf{prescriptive ethics}, which is the uniquely philosophical attempt to find the true basis of ethical thinking. Much of our discussion in the first half of this text falls under this heading since we will be examining various attempts to give an account of the basis and justification of ethical thought, belief and action.
\end{itemize}

\begin{marginfigure}
\textbf{Applied ethics}: what is the right thing to do in this
real-world case?
\end{marginfigure}

\begin{itemize}
\tightlist
\item
  How does all of this play out in real life cases? This is the concern of \textbf{applied ethics}. Under this heading are also to be found discussions of ethical issues associated with some particular area of human life, profession, or subject matter - hence medical ethics, business ethics, legal ethics, environmental ethics, bioethics and so on are sub-fields within applied ethics.
\end{itemize}

We should keep in mind as we proceed that these various regions are not always so clearly separate from one another. Our description of what people believe about ethical questions, for example, is clearly often informed by what we think they are justified in believing. Nevertheless we should keep in mind the fact that we can look at ethics from each of these different points of view and recognize that failing to do so may result in unnecessary confusion.

In conclusion we might say that philosophical ethics involves deliberately reflecting on our ideas about ethics in general and on specific applications of these ideas to actual cases and controversies. Another term for such deliberate reflection is ``critical thinking.'' This should not be looked at as a primarily negative activity as the word ``critical'' might suggest, but as the positive attempt to arrive at the truth of the matter by thinking carefully about what are often complex and ambiguous ideas and concepts. Even though, as I mentioned at the outset, all of us are equally capable of reflecting critically on our own beliefs, desires, actions and values, it does take some effort and quite a bit of practice to be able to do so effectively. This is because critical thinking is a skill like anything else that we might do with our minds (like solve algebra problems or identify different species of trees) and we shouldn't expect to be experts at it from the start. In the next chapter we will look at and get some practice using one of the most important tools for critical thinking - the logical analysis of arguments.

\hypertarget{a-little-bit-of-logic}{%
\chapter{A Little bit of logic}\label{a-little-bit-of-logic}}

\begin{epigraph}
`Contrariwise,' continued Tweedledee, `if it was so, it might be; and if
it were so, it would be; but as it isn't, it ain't. That's logic.'\\
---Lewis Carroll, Through the Looking Glass
\end{epigraph}

\begin{marginfigure}
\includegraphics{img/tenniel-tweedle-dee-dum.jpg} John Tenniel, ``Alice
meets Tweedle Dee and Tweedle Dum''
\end{marginfigure}

\newthought{Logic} is the formal study of one aspect of our use of language -- the attempt to justify or provide evidence for claims or beliefs as expressed in arguments. In this chapter we will look at the basic concepts and techniques for the logical analysis of arguments. As we will be seeing these will be very useful in our discussions of ethics since much of what we will be doing will involve careful consideration of the justification of claims we make about ethics in general as well as particular topics in ethics.

Before we get started though we need to clarify some terminology -- especially our use of the word ``argument.'' Too often this word conjures up a pointless verbal fight between people with opposed views. They argue rather than discuss because their differences of opinion are fixed in place and neither will budge. It is typically a good idea to stay away from arguments in this sense. The word argument as we are using it here, however, has quite a different meaning. For us arguments do not require differences of opinion because arguments are just attempts to explicitly provide back-up or justification for some claim that we might make. We offer arguments in this sense whenever we make the grounds for our belief explicit whether we are doing this within the confines of our own heads, in written form or spoken out loud whether or not anyone disagrees with us. Of course oftentimes arguments in this sense of the term might be given by people claiming to justify different views about some topic or other. But as we will be seeing arguments are best looked at one at a time and not necessarily in comparison with other arguments.

For philosophers, we offer arguments as attempts to provide support for whatever it is that we want to claim is true. For example, maybe we happen to believe that the death penalty is wrong, or maybe we believe the opposite. Or we might believe that the government should cut all social services programs that help the poor, or we might believe the opposite. Or we may believe that ethical principles are relative to different cultures and that there are thus no ethical principles that are universally valid; or again, we might believe the opposite. We can of course believe whatever it is that we want. That will, however, only get us so far -- either others will agree with us or not. But we can also offer reasons in support of our claims in the form of arguments. As we will be seeing, not all arguments are equally persuasive, but there are clear cut and reliable ways of evaluating them to see which really provide the support we are after and which do not.

\hypertarget{arguments-rationality-and-rhetoric}{%
\section{Arguments, Rationality and Rhetoric}\label{arguments-rationality-and-rhetoric}}

\begin{marginfigure}
\includegraphics{img/tenniel-tweedle-dee-dum.jpg} John Tenniel, ``The
Mad Hatter''
\end{marginfigure}

\newthought{Arguments} are attempts to persuade other people that they should accept the claims that we are making. Because argumentation is a method of persuasion it may seem at first glance to be similar to rhetoric, also known as ``the art of persuasion.'' People who study and practice rhetoric often claim that rational argument is just one among many different methods of persuasion, appropriate at specific times, but not fundamentally different than other methods. That is, they claim that argument is a form of rhetoric. Philosophers, on the other hand, would like to insist on the basic difference between the two. Philosophers call attention to the fact that in rhetoric:

\begin{itemize}
\item
  Appeal is made to our emotions, prejudices, fears, hopes, etc. That is, who we are and what we feel about things matters. This is its strength (advertisers use such appeals all the time) and its weakness (you will have a hard time persuading someone with lung cancer to take up smoking, because they have concerns that cigarette advertisers assume that we don't have when they try to sell us cigarettes).
\item
  Because of this, the persuasion that rhetoric produces doesn't last, once our feelings change, we are no longer convinced, and our feelings are constantly changing.
\end{itemize}

In rational argument, on the other hand:

\begin{itemize}
\item
  Appeal is made not to our emotions but to our ability to reason.
\item
  Since everyone is equally capable of reasoning, this means that arguments do not appeal to us personally. It doesn't matter who you are, a good argument will convince you.
\end{itemize}

\hypertarget{the-structure-of-arguments}{%
\section{The Structure of Arguments}\label{the-structure-of-arguments}}

\begin{marginfigure}
\includegraphics{img/tenniel-alice-queens.jpg} John Tenniel, ``Alice and
the Red and White Queens''
\end{marginfigure}

\newthought{To see} all of this more clearly, we need to take a look at how arguments work. But first we should define what we mean by an argument. An argument is a series of statements including at least one premise and a conclusion. The premises are where we start, the conclusion is where we end up. In a good argument the premises must lead us necessarily to the conclusion. More on this shortly.

Arguments are sets of statements: this means that when we are concerned with arguments we are not paying attention to many other uses of language, such as asking questions, making commands, expressing feelings. When we are offering an argument we are simply making a series of claims in which some are supposed to provide support for others. The statements that are doing the supporting, which contain the information that is the basis of our argument are known as premises. The statement that is being supported, the point of our argument is called the conclusion.

It is sometimes difficult to tell whether a set of sentences is an argument or not. Let us consider a few examples:

\begin{argument}
Parents should have the right to make decisions about their own
children.\\
Why should other people mess around in their business?\\
And please keep those lawyers out!
\end{argument}

This may seem like an argument, so how can we tell for sure? Simply by asking ourselves whether this set of sentences is a set of statements where some are supposed to support the others. So, how many statements are there here? Only one: the first sentence is a statement, the second is a question and the third is a command. In other words, even though this looks at first like an argument it is really just a single claim with no real argument given in support. It is truly amazing what people will try to get away with.

What about the next example? How many statements are in these sentences? And do any of them really offer support for any of the others?

\begin{argument}
I am convinced that aliens are living among us and you should be
convinced as well.\\
I have really good evidence for this claim.
\end{argument}

Well this is almost an argument, but not quite. There is a claim being made here: aliens are living among us. But there is no real support given for this claim, only the insistence that this person has some unknown evidence. Before we can start to evaluate this evidence to see whether it really supports the claim, we need to see it. So here we have only two separate statements without a real argument yet.

OK, none of these sets of sentences have yet been real arguments even if they might have seemed to be at first glance. Now consider the following example:

\begin{argument}
Christopher Columbus was a criminal, because anyone who kills innocent
people, kidnaps others, and steals their valuables is a criminal and
that is just what he did.
\end{argument}

Here the grammatical form is a little misleading. This is an argument in spite of the fact that there is only one sentence. Why? Because this one sentence expresses a few different claims and some of these claims are offered as supports for others. We can see this if we break it up into individual claims and change the order around like so:

\begin{argument}
Anyone who kills people, kidnaps other people and steals their valuables
is a criminal.\\
Christopher Columbus did all of these things.\\

So Christopher Columbus was a criminal.
\end{argument}

Perhaps this is not yet a very convincing argument, but at least it is an argument unlike the first examples.

It is not always so clear which statements in an argument are the premises and which statement is the conclusion. Often, but not always, these are signaled with one of a number of typical words or phrases that function as premise or conclusion indicators. Paying attention to these typical words and phrases can help you to disentangle the argument from the peculiarities of a writer's style.

\hypertarget{premises}{%
\subsection*{premises}\label{premises}}
\addcontentsline{toc}{subsection}{premises}

It is often the case that arguments are presented with the conclusion first to emphasize to the audience where things are leading. The following common words are often used to indicate what is playing the logical role of conclusion.

\begin{itemize}
\tightlist
\item
  Because
\item
  Since
\item
  In light of the fact that
\item
  In view of the following evidence
\end{itemize}

This is not an exhaustive list. Basically, when reading an argument you can pick out the premises by asking yourself where the writer is starting from and where he or she is going. The first is the set of premises and the second is the conclusion.

\hypertarget{conclusions}{%
\subsection*{conclusions}\label{conclusions}}
\addcontentsline{toc}{subsection}{conclusions}

\begin{itemize}
\tightlist
\item
  Therefore
\item
  It follows that
\item
  Thus
\item
  It should be clear that
\end{itemize}

These words and phrases indicate that this is where the writer (or speaker) is going with the argument. Notice that in many actual arguments the conclusion is given first, as when a lawyer begins her argument in court with the claim, ``Your honor, ladies and gentlemen of the jury, my client is not guilty,'' and then goes on to present the evidence. For the sake of analyzing an argument philosophers like to clarify its logical structure by writing it in standard form.

\hypertarget{pattern-of-reasoning}{%
\subsection*{pattern of reasoning}\label{pattern-of-reasoning}}
\addcontentsline{toc}{subsection}{pattern of reasoning}

One other thing to watch for when looking at arguments is words and phrases that indicate the structure of the reasoning itself. These are ways of pointing out exactly how the premises are supposed to support the conclusion, indicators of the pattern or form of reasoning involved.

\begin{itemize}
\tightlist
\item
  because of these, that has to be true
\item
  if this then that, otherwise\ldots{}
\item
  all of the above so\ldots{}
\item
  this is the only option that works
\item
  if we assume that this is true we get a ridiculous result so it can't be true
\end{itemize}

These indicate the general logic form of argument being followed. Is it a matter of necessity, other conditions present or absent, summation of influences, or a process of elimination, or are we showing something indirectly by showing that denying it makes no sense? The more formal study of logic looks carefully at these and many other different patterns of reasoning.

\hypertarget{validity-and-soundness}{%
\section{Validity and Soundness}\label{validity-and-soundness}}

\begin{marginfigure}
\includegraphics{img/tenniel-alice-humpty.jpg} John Tenniel, ``Alice
Meets Humpty''
\end{marginfigure}

\newthought{Not all} arguments are equal. Just because we have an arbitrary set of premises supporting some random conclusion doesn't mean that we should feel ourselves compelled to buy this conclusion. Instead, as we will see, there are some arguments that really are better than others. Really good arguments are compelling on their own, and we should, as long as we are being rational, have no choice but to accept them. For the skeptics out there who doubt that we will ever be able to create such an argument, I should also point out that the clearest and best arguments really don't end up saying anything very controversial or extraordinary. This is one of the limitations that logic imposes on us: if we are really being logical and using only reliable arguments we may have to refrain from claiming to be able to establish very much. Understanding the logic of arguments, if nothing else, should encourage us to be a little more modest in our claims to knowledge.

When we are arguing what we are doing is trying to establish the truth of something that we don't know on the basis of other things that we already know or accept. What we are interested in is establishing the truth of the conclusion, yet for some reason it's truth is not obviously apparent to us so we need to establish it on the basis of other claims the truth of which we can already accept. Arguments move us from the known to the unknown.

To take a simple example, suppose we would like to establish that Socrates fears death. We don't have any direct reason for thinking that this is true. But we do know some other things that may be of use in establishing this. First we know that Socrates is a human being. Second we know that all human beings are mortal. Third, we know that all mortals fear death. In standard form this would be arranged like so:

\begin{argument}
Socrates is a human being.\\
All human beings are mortal.\\
All mortals fear death.\\

So Socrates fears death.
\end{argument}

\hypertarget{key-concepts}{%
\subsection*{Key Concepts}\label{key-concepts}}
\addcontentsline{toc}{subsection}{Key Concepts}

The information in the premises is enough information, as we can easily see, to establish our conclusion. Since Socrates is human he must be mortal, and he must fear death, since all mortals fear death. This argument seems like a pretty solid piece of reasoning. But how can we tell in general whether an argument is a good argument? It turns out that there are two questions we will need to ask about an argument in order to determine whether or not it is a good argument: \marginnote{VALIDITY: in a valid argument IF the premises are true the conclusion MUST also be true.}

\begin{itemize}
\item
  Is there a clear and solid connection between every step of the reasoning that leads us inevitably from premises to conclusion? In philosophical terminology: \textbf{is it valid}?
\item
  Are the claims that we started from, our premises, really true? In philosophical terminology: \textbf{is it sound}?
\end{itemize}

\marginnote{SOUNDNESS: A sound argument is a valid argument that also has TRUE premises.}

How do we answer these questions for the example above? It seems that there is a clear and solid connection between what the premises are saying and what the conclusion is saying. In fact we already showed this when showed that the conclusion necessarily follows from the premises. Technically this is a short and informal proof of its strength as an argument, that is, of its validity. So the answer to the first question is, yes, it is valid.

As far as the second question goes, however, we may have our doubts. Are all of the premises really true? Socrates is (or was) a human being -- he was one of the first philosophers. And all human beings are in fact mortal. But do we really know whether all mortals, past, present and future fear death? So here is the one small weakness of the argument. If we could be assured that this premise was true the argument would be completely convincing and would provide adequate backup for the conclusion. But it rests, unfortunately, on a weak premise, so it is not a sound argument.

One thing to notice here is that the test for validity is entirely independent of the test for soundness. It is a little misleading, as we can now see, to ask whether arguments are either good or bad. More precisely, they can be:

\begin{itemize}
\item
  \emph{Valid and sound}: these are the best arguments, because the premises really establish the conclusion, and the premises are true -- hence the conclusion really is true.
\item
  \emph{Valid but not sound}: these are promising arguments that exhibit good logical form, but that rely on less than perfect information in their premises, and so are not completely solid.
\item
  \emph{Invalid}: these arguments are bad arguments since they do not establish what they claim to be establishing. All invalid arguments are automatically unsound, since sound arguments are a subset of valid arguments.
\end{itemize}

\hypertarget{more-examples}{%
\subsection*{More examples}\label{more-examples}}
\addcontentsline{toc}{subsection}{More examples}

Learning how to identify valid arguments is important for a course in philosophical ethics, since the philosophical approach to ethics consists largely of the examination of arguments about ethical issues. And the best way to learn this is by practicing. Consider the following argument, conveniently written in standard form:

\begin{argument}
The earth is a rotating sphere moving around the sun.\\
We are all on the surface of the earth.\\
Anything on the surface of a moving object moves with that object.\\

So we are all moving around the sun.
\end{argument}

Forget for a moment about whether or not you buy the conclusion on its own. In analyzing an argument we need to know whether the premises support the conclusion adequately, so we pretend that we are not sure about the truth of the conclusion. Our first test is the test of validity. We ask ourselves: if the premises were true, could the conclusion be otherwise? Is the truth of the conclusion guaranteed by the truth of the premises? In this case it seems clear that if we are in fact all on the surface of an object that is moving around the sun, then we would all also have to be moving around the sun. So the argument is valid.

Notice that establishing an argument's validity is not yet establishing that the conclusion is really true. It is only establishing that the conclusion would be true, if only we could show that the premises were true. In fact this argument was rejected until about 500 years ago because nobody was willing to accept the truth of the first premise. Establishing that this was true took quite a bit of effort by Copernicus, Kepler, Galileo and other early modern scientists. However, we now know that the premises are true. So this argument is not only valid, but also sound. And since it is sound we have proven beyond the shadow of a doubt that the conclusion is true. One more thing to point out here is that this argument has always been sound (or at least as long as the solar system has existed) even if many people denied the truth of the first premise. They were simply mistaken in this denial.

Let's look at another example:

\begin{argument}
If you want to see the world, you should join the navy.\\
Jane wants to see the world.\\

So Jane should join the navy.
\end{argument}

This argument is a little trickier because it contains an IF -- THEN statement. IF -- THEN statements, also known as conditionals, make indirect claims. They don't just tell us what is the case, they tell us what would be the case if, or on condition that, something else were true. With this in mind let us consider this second argument. First we check for validity, by assuming that the premises are true and seeing if the conclusion would have to be true as well. In other words we are not yet interested in whether or not they really are true, but whether the argument works as an argument, whether the conclusion logically follows from the premises. It seems pretty clear, or at least it should seem clear, that this argument is valid. This is because if, as the first premise claims, the navy really is the best way to see the world, and if as the second premise claims, a person named Jane wants to see the world, then she should clearly join the navy. Notice that this argument's validity does not have anything to do with its content, with the particular claims being made. Instead, validity is a matter of form, so that we could substitute any other content for the content of this argument without affecting its validity. Essentially this argument has the following form:

\begin{argument}
If you want to do A, then you should do B.\\
Person P wants to do A.\\

So person P should do B.
\end{argument}

Here A, B can be substituted by any statements we please, and P can be any person we please, as long as our substitution is consistent throughout the argument. In all cases the resulting argument will turn out to be valid. Try it and you will see that the resulting arguments all come out valid. This is because validity is a matter of logical form regardless of the content we are arguing about.

The soundness of arguments, however, unlike validity, has everything to do with content, because an argument is sound when it is valid and it also has true premises. Back to the argument about Jane. Is it sound? First we note that it is valid, then we ask whether or not the premises are really true. Consider the first premise: ``If you want to see the world you should join the navy.'' It may be true that joining the navy is one way to see the world (provided that you don't end up on a submarine, or in the engine room of a ship), but is it the only way? Of course not, so the first premise is just false. The second premise is also questionable, but for a different reason -- we simply do not know who Jane is since this is a fictional example. So in spite of its validity this argument is unsound and we need not accept the conclusion as a true statement. It may in fact be true, but this argument gives us no good reason for thinking so. As an exercise you might want to try coming up with a sound argument that follows the form of this one.

Now consider, as our next example, the following argument:

\begin{argument}
If you want to see the world, you should join the navy.\\
Jane joined the navy.\\

So Jane wants to see the world.
\end{argument}

This argument seems similar to the previous one, but it has one important difference. The conclusion of this argument was the second premise of the last argument, and the second premise of this argument was its conclusion. What happens to the validity of the argument when we make this simple change? Notice what this argument is saying. It is offering an explanation of why it is that Jane joined the navy -- because she wanted to see the world. The question is, and this is the way we check for validity, are there any other possible explanations of why she joined the navy that are consistent with the premises? In other words, is it possible for the premises to both be true and the conclusion false? The answer is yes. It all hinges on what the first premise doesn't say. It doesn't say that the only possible reason to join the navy is the desire to see the world. It just says that if that's what you happen to want then the navy is for you. So Jane could have joined the navy only because she wanted to learn all there is to know about marine diesel engines without caring whether she learned this in New Jersey or in the South Pacific ocean. To put this in yet another way: if it is at all possible, if there are no contradictions involved, for the premises of an argument to be true and the conclusion false, then the argument is invalid. This argument is invalid for precisely this reason. Furthermore, since it is invalid, this automatically makes it unsound, since in order for it to be sound it has to first be valid.

\hypertarget{proof-and-counterexamples}{%
\section{Proof and Counterexamples}\label{proof-and-counterexamples}}

\hypertarget{find-out-more}{%
\section*{Find out more}\label{find-out-more}}
\addcontentsline{toc}{section}{Find out more}

\hypertarget{fallacies-and-biases}{%
\chapter{Fallacies and Biases}\label{fallacies-and-biases}}

\hypertarget{fallacies-of-relevance}{%
\section{Fallacies of Relevance}\label{fallacies-of-relevance}}

\hypertarget{fallacies-of-ambiguity}{%
\section{Fallacies of Ambiguity}\label{fallacies-of-ambiguity}}

\hypertarget{fallacies-of-presumption}{%
\section{Fallacies of Presumption}\label{fallacies-of-presumption}}

\hypertarget{cognitive-biases}{%
\section{Cognitive Biases}\label{cognitive-biases}}

\hypertarget{find-out-more-1}{%
\section*{Find out more}\label{find-out-more-1}}
\addcontentsline{toc}{section}{Find out more}

\begin{itemize}
\tightlist
\item
  \href{https://yourlogicalfallacyis.com/}{Your Fallacy Is} is a nicely designed website with more examples of common fallacies.

  \begin{itemize}
  \tightlist
  \item
    \url{https://yourlogicalfallacyis.com/}
  \end{itemize}
\item
  \href{https://yourbias.is/}{Your Bias Is}: is the sister site to Your Fallacy Is and focuses on common cognitive biases.

  \begin{itemize}
  \tightlist
  \item
    \url{https://yourbias.is/}
  \end{itemize}
\end{itemize}

\hypertarget{part-thinking-about-ethics}{%
\part{Thinking About Ethics}\label{part-thinking-about-ethics}}

\hypertarget{relativism}{%
\chapter{Relativism}\label{relativism}}

\hypertarget{claims-and-consequences}{%
\section{Claims and Consequences}\label{claims-and-consequences}}

\hypertarget{defending-relativism}{%
\section{Defending Relativism}\label{defending-relativism}}

\hypertarget{moving-on}{%
\section{Moving on}\label{moving-on}}

\hypertarget{religion-and-ethics}{%
\chapter{Religion and Ethics}\label{religion-and-ethics}}

\hypertarget{divine-command-theory}{%
\section{Divine Command Theory}\label{divine-command-theory}}

\hypertarget{a-nasty-dilemma}{%
\section{A Nasty Dilemma}\label{a-nasty-dilemma}}

\hypertarget{natural-law-theory}{%
\section{Natural Law Theory}\label{natural-law-theory}}

\hypertarget{human-nature}{%
\section{Human Nature?}\label{human-nature}}

\hypertarget{egoism}{%
\chapter{Egoism}\label{egoism}}

\hypertarget{whats-in-it-for-me}{%
\section{What's in it for me?}\label{whats-in-it-for-me}}

\hypertarget{psychological-egoism}{%
\section{Psychological Egoism}\label{psychological-egoism}}

\hypertarget{ethical-egoism}{%
\section{Ethical Egoism}\label{ethical-egoism}}

\hypertarget{capitalism-and-the-common-good}{%
\section{Capitalism and the common good}\label{capitalism-and-the-common-good}}

\hypertarget{the-social-contract}{%
\chapter{The Social Contract}\label{the-social-contract}}

\hypertarget{hobbes-and-the-invention-of-society}{%
\section{Hobbes and the invention of society}\label{hobbes-and-the-invention-of-society}}

\hypertarget{the-prisoners-dilemma}{%
\section{The Prisoners Dilemma}\label{the-prisoners-dilemma}}

\hypertarget{why-should-we-follow-the-rules}{%
\section{Why should we follow the rules?}\label{why-should-we-follow-the-rules}}

\hypertarget{utilitarianism}{%
\chapter{Utilitarianism}\label{utilitarianism}}

\hypertarget{the-highest-good}{%
\section{The highest good}\label{the-highest-good}}

\hypertarget{why-should-i-care}{%
\section{Why should I care?}\label{why-should-i-care}}

\hypertarget{problems-problems}{%
\section{Problems, problems}\label{problems-problems}}

\hypertarget{kant-and-the-ethics-of-duty}{%
\chapter{Kant and the ethics of duty}\label{kant-and-the-ethics-of-duty}}

\hypertarget{what-do-we-owe-one-another}{%
\section{What do we owe one another?}\label{what-do-we-owe-one-another}}

\hypertarget{the-categorical-imperative}{%
\section{The Categorical Imperative}\label{the-categorical-imperative}}

\hypertarget{rights-and-duties}{%
\section{Rights and duties}\label{rights-and-duties}}



\end{document}
