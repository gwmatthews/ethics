\documentclass[]{memoir}
\usepackage{lmodern}
\usepackage{amssymb,amsmath}
\usepackage{ifxetex,ifluatex}
\usepackage{fixltx2e} % provides \textsubscript
\ifnum 0\ifxetex 1\fi\ifluatex 1\fi=0 % if pdftex
  \usepackage[T1]{fontenc}
  \usepackage[utf8]{inputenc}
\else % if luatex or xelatex
  \ifxetex
    \usepackage{mathspec}
  \else
    \usepackage{fontspec}
  \fi
  \defaultfontfeatures{Ligatures=TeX,Scale=MatchLowercase}
\fi
% use upquote if available, for straight quotes in verbatim environments
\IfFileExists{upquote.sty}{\usepackage{upquote}}{}
% use microtype if available
\IfFileExists{microtype.sty}{%
\usepackage{microtype}
\UseMicrotypeSet[protrusion]{basicmath} % disable protrusion for tt fonts
}{}
\usepackage{hyperref}
\hypersetup{unicode=true,
            pdftitle={Philosophical Ethics: A Guidebook for Beginners},
            pdfauthor={George Matthews},
            pdfborder={0 0 0},
            breaklinks=true}
\urlstyle{same}  % don't use monospace font for urls
\usepackage{natbib}
\bibliographystyle{apalike}
\usepackage{longtable,booktabs}
\usepackage{graphicx}
% grffile has become a legacy package: https://ctan.org/pkg/grffile
\IfFileExists{grffile.sty}{%
\usepackage{grffile}
}{}
\makeatletter
\def\maxwidth{\ifdim\Gin@nat@width>\linewidth\linewidth\else\Gin@nat@width\fi}
\def\maxheight{\ifdim\Gin@nat@height>\textheight\textheight\else\Gin@nat@height\fi}
\makeatother
% Scale images if necessary, so that they will not overflow the page
% margins by default, and it is still possible to overwrite the defaults
% using explicit options in \includegraphics[width, height, ...]{}
\setkeys{Gin}{width=\maxwidth,height=\maxheight,keepaspectratio}
\IfFileExists{parskip.sty}{%
\usepackage{parskip}
}{% else
\setlength{\parindent}{0pt}
\setlength{\parskip}{6pt plus 2pt minus 1pt}
}
\setlength{\emergencystretch}{3em}  % prevent overfull lines
\providecommand{\tightlist}{%
  \setlength{\itemsep}{0pt}\setlength{\parskip}{0pt}}
\setcounter{secnumdepth}{5}
% Redefines (sub)paragraphs to behave more like sections
\ifx\paragraph\undefined\else
\let\oldparagraph\paragraph
\renewcommand{\paragraph}[1]{\oldparagraph{#1}\mbox{}}
\fi
\ifx\subparagraph\undefined\else
\let\oldsubparagraph\subparagraph
\renewcommand{\subparagraph}[1]{\oldsubparagraph{#1}\mbox{}}
\fi

%%% Use protect on footnotes to avoid problems with footnotes in titles
\let\rmarkdownfootnote\footnote%
\def\footnote{\protect\rmarkdownfootnote}

%%% Change title format to be more compact
\usepackage{titling}

% Create subtitle command for use in maketitle
\providecommand{\subtitle}[1]{
  \posttitle{
    \begin{center}\large#1\end{center}
    }
}

\setlength{\droptitle}{-2em}

  \title{Philosophical Ethics: A Guidebook for Beginners}
    \pretitle{\vspace{\droptitle}\centering\huge}
  \posttitle{\par}
    \author{George Matthews}
    \preauthor{\centering\large\emph}
  \postauthor{\par}
      \predate{\centering\large\emph}
  \postdate{\par}
    \date{2019-11-22}

\usepackage{booktabs}
\usepackage{amsthm}
\makeatletter
\def\thm@space@setup{%
  \thm@preskip=8pt plus 2pt minus 4pt
  \thm@postskip=\thm@preskip
}
\makeatother

\begin{document}
\maketitle

{
\setcounter{tocdepth}{1}
\tableofcontents
}
\hypertarget{preface}{%
\chapter*{Preface}\label{preface}}
\addcontentsline{toc}{chapter}{Preface}

This is a an open source textbook in development.

\textbf{License CC BY-SA 4.0 license}

The book is released under a creative commons \href{https://creativecommons.org/licenses/by-sa/4.0/}{CC BY-SA 4.0} license. This means that this book can be reused, remixed, retained, revised and redistributed (including commercially) as long as appropriate credit is given to the authors. If you remix, or modify the original version of this open textbook, you must redistribute all versions of this open textbook under the same license - CC BY-SA 4.0.

My remix is based on the work of Matthew J. C. Crump (2018). Open tools for writing open interactive textbooks (and more). \url{https://crumplab.github.io/programmingforpsych/}

\hypertarget{part-some-preliminaries}{%
\part{Some Preliminaries}\label{part-some-preliminaries}}

\hypertarget{the-examined-life}{%
\chapter{The Examined Life}\label{the-examined-life}}

We humans are endowed with a unique capacity, the ability to reflect on what we believe and do. Unlike other animals, we are capable of taking a distance from the evidence of our senses and asking ourselves, ``Should I trust what I see or not?'' Likewise in the realm of desire and action: we can examine our own desires, intentions and plans and ask ourselves, ``Should I act on these or not?'' In both cases we are capable of stepping back from the immediate demands of our situation and seeking orientation from another source - we seek \emph{reasons} to believe or doubt what we see and reasons to follow or resist our urges. This is worth dwelling on for a little while since it doesn't seem to me to be too much of an exaggeration to say that much of our knowledge of and power over the natural world as well as many of the dilemmas we face in acting in the world are made possible by the fact of our reflective capacity to respond to reasons.

\hypertarget{what-do-i-know}{%
\section{What do I know?}\label{what-do-i-know}}

As mammals outfitted with complex nervous systems we are constantly receiving input from our senses. We perceive colors, sounds, smells, tastes and bodily sensations from the moment we wake up until the moment we fall completely asleep each day and even in sleep we are not completely shut off from sensory input. All of this raw sensation is processed behind the scenes of our conscious minds in myriad ways we are just beginning to understand, and as a result we are presented with a picture of a world of objects in space and time interacting with each other and us in many different ways. This much we share with other animals, or at least with those with whom we have a cerebral cortex in common. But unlike those other animals, our unique linguistic capacity enables us to engage with the world of our experience in a way that they cannot.

First of all, our ability to use and understand language enables us to explicitly categorize and classify what we experience - things are not just there in our surroundings as objects we happen to come across. Instead we both consciously and unconsciously organize the things we encounter into groups based on concepts such as: edible or inedible, animate or inanimate, threatening or safe, members of our social group or not members of our group, male or female, cause or effect, and so on. Now although there is clear evidence that certain animals do this to a limited extent as well - dogs, for example distinguish between their owners and strangers quite readily and reliably - for us, these acts of categorizing and classifying can be endlessly expanded and modified and also made fully explicit to our own awareness. We can endlessly expand our categories to include anything and everything conceivable - ``80's hair bands,'' ``things not good for eating in bed,'' ``mops,'' and ``former presidents,'' just to give a quick random sample. Not only do we thus have an infinitely variable way of looking at the world and organizing our experience according to concepts and categories but we can see and understand how we are doing so since we can articulate this organization verbally. Hence we can recognize that we are dividing up the world in a particular way and do it in a different way. We can add new categories or modify how we use them as we notice new similarities or subtle distinctions among things. We may also revise and refine our categories as accumulated personal and shared experience reveals to us their strengths and weaknesses - whether they ``carve nature at its joints'' or not. This ability to look at things in new ways as a result of collectively accumulated experience is rooted in the fact that we use language to do so and language is both infinitely extensible and essentially shared with other humans. Most importantly for the story I am telling here, we can ask ourselves about the implications of the way we look at the world, and we can wonder about whether we have good reasons for looking at things as we do.

This is a point that it is hard to overemphasize but also easy to miss since we take it so much for granted. By asking ourselves about the reasons we have for believing that some aspect or other of our experience is true we are asking ourselves not only about the way things are, but about the way things should be; not just what we happen to believe about things based on their appearance to us, but about what we should believe about them because it reflects their true reality. And by asking ourselves such questions we are asking what philosophers call normative questions, questions that have to do with values, with concepts like right, wrong, good, bad, true, false, beautiful and ugly.

\hypertarget{what-should-i-do}{%
\section{What should I do?}\label{what-should-i-do}}

Thus far I have been emphasizing the role of reflection and the seeking of reasons in our attempts to understand the world in which we live. But this of course is an incomplete picture, since we are not just disembodied minds looking at and trying to figure out the world. We are embodied, social beings who feel and act on needs and impulses, experience emotions, form and try to realize intentions, coordinate or compete with others, and seek or shun each others' company. This practical side of human life is, just as we have seen for our experiential selves, equally capable of being made explicit and becoming an object of our reflective capacities. We don't have to simply act on whatever urges happen to come to our attention, we can stop and think about what to do instead. Hence we find ourselves presented with choices - should I follow my immediate urges, or should I refrain from doing so in order to realize other goals? Once again, just as in the case of belief, our reflective capacity introduces a normative dimension to human life as we come to ask ourselves questions about our own needs, desires and decisions. We wonder what we should do in some particular situation, perhaps when our feelings are telling us one thing and our experience is reminding us of the bad results the last time we acted on similar impulses. And this generalizes as well, as we come to reflect on our motivations as such, on which of our goals are more worth pursuing in the long run, on the nature of human motivation and goals in general, on what might truly be the best way to live our lives. And thus philosophical ethics is born as a product of reflection on our own decision making as potentially thoughtful social animals.

\hypertarget{philosophical-ethics}{%
\section{Philosophical Ethics}\label{philosophical-ethics}}

Philosophical ethics is nothing but the deliberate pursuit and clarification of this kind of reflection on our own values, actions and decisions. Even though, as I have been emphasizing, we all have the capacity to reflect on our lives and choices, we do not always spend the time or make the effort to do this carefully and deeply. This is because we are mostly preoccupied with the practical details of our own lives. We are too busy living to take the time to stop and think about the significance of what we are doing. However, at times in the lives of both individuals and societies the need to reflect more clearly on what we are doing becomes more of an imperative. For individuals the need to stop and think and to reconsider the basic assumptions on which we act often arises in relation to important life events or radical changes - the sudden loss of a loved one; the birth of a child; living through a natural disaster or a war; or even the transition to adulthood in which one assumes full moral and legal responsibility while also gaining the full rights and privileges accorded to adults. These are topics and situations, as we will see later, that are often the focus of discussions in the branch of philosophical ethics called applied ethics. In the case of societies, philosophical thinking in general and philosophical ethics in particular likewise flourish in times of great stress or change - for example when radically different societies suddenly make contact with each other; when new groups and ways of living displace old groups and ways; when new discoveries challenge peoples' basic views of the nature of things; when societies find their very existence threatened by seemingly insurmountable obstacles. In cases like these it becomes imperative to reflect more carefully on what we assume is of value to us individually and as a society, on what counts as a good life.

Philosophical ethics or moral philosophy is concerned with a number of different sorts of questions about ethics. Thus the broader field of ethics can be divided up into a number of different regions or areas of concern. Some of the main questions and their corresponding sub-fields are:

\begin{itemize}
\tightlist
\item
  What ethical views do real people have? This is the concern of descriptive ethics, which tries to figure out what beliefs people happen to actually have concerning ethical questions. As such, descriptive ethics is not exclusively a philosophical approach to ethics in that sociologists, psychologists, anthropologists and other social scientists are also concerned with people's ethical beliefs in this sense.
\item
  What is the nature of ethical thinking and ethical concepts? This is usually referred to as meta-ethics, which refers to a higher-order or ``meta-level'' discussion about ethical modes of thinking. Here again, philosophers as well as social scientists often ask meta-ethical questions in their attempts to understand what is distinctive about ethical thinking as opposed to other modes of cognition.
\item
  What ethical principles or decision-making procedures are really justified? This is the basic question addressed by prescriptive ethics, which is the uniquely philosophical attempt to find the true basis of ethical thinking. Much of our discussion in the first half of this text falls under this heading since we will be examining various attempts to give an account of the basis and justification of ethical thought, belief and action.
\item
  How does all of this play out in real life cases? This is the concern of applied ethics. Under this heading are also to be found discussions of ethical issues associated with some particular area of human life, profession, or subject matter - hence medical ethics, business ethics, legal ethics, environmental ethics, bioethics and so on are sub-fields within applied ethics.
\end{itemize}

We should keep in mind as we proceed that these various regions are not always so clearly separate from one another. Our description of what people believe about ethical questions, for example, is clearly often informed by what we think they are justified in believing. Nevertheless we should keep in mind the fact that we can look at ethics from each of these different points of view and recognize that failing to do so may result in unnecessary confusion.

In conclusion we might say that philosophical ethics involves deliberately reflecting on our ideas about ethics in general and on specific applications of these ideas to actual cases and controversies. Another term for such deliberate reflection is ``critical thinking.'' This should not be looked at as a primarily negative activity as the word ``critical'' might suggest, but as the positive attempt to arrive at the truth of the matter by thinking carefully about what are often complex and ambiguous ideas and concepts. Even though, as I mentioned at the outset, all of us are equally capable of reflecting critically on our own beliefs, desires, actions and values, it does take some effort and quite a bit of practice to be able to do so effectively. This is because critical thinking is a skill like anything else that we might do with our minds (like solve algebra problems or identify different species of trees) and we shouldn't expect to be experts at it from the start. In the next chapter we will look at and get some practice using one of the most important tools for critical thinking - the logical analysis of arguments.

\hypertarget{a-little-bit-of-logic}{%
\chapter{A Little bit of logic}\label{a-little-bit-of-logic}}

\hypertarget{arguments-rationality-and-rhetoric}{%
\section{Arguments, Rationality and Rhetoric}\label{arguments-rationality-and-rhetoric}}

testing

\hypertarget{validity-and-soundness}{%
\section{Validity and Soundness}\label{validity-and-soundness}}

\hypertarget{proof-and-counterexamples}{%
\section{Proof and Counterexamples}\label{proof-and-counterexamples}}

\hypertarget{fallacies-and-biases}{%
\chapter{Fallacies and Biases}\label{fallacies-and-biases}}

\hypertarget{fallacies-of-relevance}{%
\section{Fallacies of Relevance}\label{fallacies-of-relevance}}

\hypertarget{fallacies-of-ambiguity}{%
\section{Fallacies of Ambiguity}\label{fallacies-of-ambiguity}}

\hypertarget{fallacies-of-presumption}{%
\section{Fallacies of Presumption}\label{fallacies-of-presumption}}

\hypertarget{cognitive-biases}{%
\section{Cognitive Biases}\label{cognitive-biases}}

\hypertarget{more}{%
\chapter{More}\label{more}}

\hypertarget{your-fallacy-is}{%
\section{Your Fallacy Is}\label{your-fallacy-is}}

\hypertarget{your-bias-is}{%
\section{Your Bias Is}\label{your-bias-is}}

\hypertarget{part-thinking-about-ethics}{%
\part{Thinking About Ethics}\label{part-thinking-about-ethics}}

\hypertarget{relativism}{%
\chapter{Relativism}\label{relativism}}

\hypertarget{claims-and-consequences}{%
\section{Claims and Consequences}\label{claims-and-consequences}}

\hypertarget{defending-relativism}{%
\section{Defending Relativism}\label{defending-relativism}}

\hypertarget{moving-on}{%
\section{Moving on}\label{moving-on}}

\hypertarget{religion-and-ethics}{%
\chapter{Religion and Ethics}\label{religion-and-ethics}}

\hypertarget{divine-command-theory}{%
\section{Divine Command Theory}\label{divine-command-theory}}

\hypertarget{a-nasty-dilemma}{%
\section{A Nasty Dilemma}\label{a-nasty-dilemma}}

\hypertarget{natural-law-theory}{%
\section{Natural Law Theory}\label{natural-law-theory}}

\hypertarget{human-nature}{%
\section{Human Nature?}\label{human-nature}}

\hypertarget{egoism}{%
\chapter{Egoism}\label{egoism}}

\hypertarget{whats-in-it-for-me}{%
\section{What's in it for me?}\label{whats-in-it-for-me}}

\hypertarget{psychological-egoism}{%
\section{Psychological Egoism}\label{psychological-egoism}}

\hypertarget{ethical-egoism}{%
\section{Ethical Egoism}\label{ethical-egoism}}

\hypertarget{capitalism-and-the-common-good}{%
\section{Capitalism and the common good}\label{capitalism-and-the-common-good}}

\hypertarget{even-more}{%
\chapter{Even More}\label{even-more}}

\hypertarget{the-social-contract}{%
\chapter{The Social Contract}\label{the-social-contract}}

\hypertarget{hobbes-and-the-invention-of-society}{%
\section{Hobbes and the invention of society}\label{hobbes-and-the-invention-of-society}}

\hypertarget{the-prisoners-dilemma}{%
\section{The Prisoners Dilemma}\label{the-prisoners-dilemma}}

\hypertarget{why-should-we-follow-the-rules}{%
\section{Why should we follow the rules?}\label{why-should-we-follow-the-rules}}

\hypertarget{utilitarianism}{%
\chapter{Utilitarianism}\label{utilitarianism}}

\hypertarget{the-highest-good}{%
\section{The highest good}\label{the-highest-good}}

\hypertarget{why-should-i-care}{%
\section{Why should I care?}\label{why-should-i-care}}

\hypertarget{problems-problems}{%
\section{Problems, problems}\label{problems-problems}}

\hypertarget{kant-and-the-ethics-of-duty}{%
\chapter{Kant and the ethics of duty}\label{kant-and-the-ethics-of-duty}}

\hypertarget{what-do-we-owe-one-another}{%
\section{What do we owe one another?}\label{what-do-we-owe-one-another}}

\hypertarget{the-categorical-imperative}{%
\section{The Categorical Imperative}\label{the-categorical-imperative}}

\hypertarget{rights-and-duties}{%
\section{Rights and duties}\label{rights-and-duties}}

\hypertarget{and-still-more}{%
\chapter{And Still More}\label{and-still-more}}

\bibliography{book.bib,packages.bib}


\end{document}
